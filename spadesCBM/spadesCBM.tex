\documentclass[]{article}
\usepackage{lmodern}
\usepackage{amssymb,amsmath}
\usepackage{ifxetex,ifluatex}
\usepackage{fixltx2e} % provides \textsubscript
\ifnum 0\ifxetex 1\fi\ifluatex 1\fi=0 % if pdftex
  \usepackage[T1]{fontenc}
  \usepackage[utf8]{inputenc}
\else % if luatex or xelatex
  \ifxetex
    \usepackage{mathspec}
  \else
    \usepackage{fontspec}
  \fi
  \defaultfontfeatures{Ligatures=TeX,Scale=MatchLowercase}
\fi
% use upquote if available, for straight quotes in verbatim environments
\IfFileExists{upquote.sty}{\usepackage{upquote}}{}
% use microtype if available
\IfFileExists{microtype.sty}{%
\usepackage{microtype}
\UseMicrotypeSet[protrusion]{basicmath} % disable protrusion for tt fonts
}{}
\usepackage[margin=1in]{geometry}
\usepackage{hyperref}
\hypersetup{unicode=true,
            pdftitle={spadesCBM},
            pdfborder={0 0 0},
            breaklinks=true}
\urlstyle{same}  % don't use monospace font for urls
\usepackage{color}
\usepackage{fancyvrb}
\newcommand{\VerbBar}{|}
\newcommand{\VERB}{\Verb[commandchars=\\\{\}]}
\DefineVerbatimEnvironment{Highlighting}{Verbatim}{commandchars=\\\{\}}
% Add ',fontsize=\small' for more characters per line
\usepackage{framed}
\definecolor{shadecolor}{RGB}{248,248,248}
\newenvironment{Shaded}{\begin{snugshade}}{\end{snugshade}}
\newcommand{\KeywordTok}[1]{\textcolor[rgb]{0.13,0.29,0.53}{\textbf{#1}}}
\newcommand{\DataTypeTok}[1]{\textcolor[rgb]{0.13,0.29,0.53}{#1}}
\newcommand{\DecValTok}[1]{\textcolor[rgb]{0.00,0.00,0.81}{#1}}
\newcommand{\BaseNTok}[1]{\textcolor[rgb]{0.00,0.00,0.81}{#1}}
\newcommand{\FloatTok}[1]{\textcolor[rgb]{0.00,0.00,0.81}{#1}}
\newcommand{\ConstantTok}[1]{\textcolor[rgb]{0.00,0.00,0.00}{#1}}
\newcommand{\CharTok}[1]{\textcolor[rgb]{0.31,0.60,0.02}{#1}}
\newcommand{\SpecialCharTok}[1]{\textcolor[rgb]{0.00,0.00,0.00}{#1}}
\newcommand{\StringTok}[1]{\textcolor[rgb]{0.31,0.60,0.02}{#1}}
\newcommand{\VerbatimStringTok}[1]{\textcolor[rgb]{0.31,0.60,0.02}{#1}}
\newcommand{\SpecialStringTok}[1]{\textcolor[rgb]{0.31,0.60,0.02}{#1}}
\newcommand{\ImportTok}[1]{#1}
\newcommand{\CommentTok}[1]{\textcolor[rgb]{0.56,0.35,0.01}{\textit{#1}}}
\newcommand{\DocumentationTok}[1]{\textcolor[rgb]{0.56,0.35,0.01}{\textbf{\textit{#1}}}}
\newcommand{\AnnotationTok}[1]{\textcolor[rgb]{0.56,0.35,0.01}{\textbf{\textit{#1}}}}
\newcommand{\CommentVarTok}[1]{\textcolor[rgb]{0.56,0.35,0.01}{\textbf{\textit{#1}}}}
\newcommand{\OtherTok}[1]{\textcolor[rgb]{0.56,0.35,0.01}{#1}}
\newcommand{\FunctionTok}[1]{\textcolor[rgb]{0.00,0.00,0.00}{#1}}
\newcommand{\VariableTok}[1]{\textcolor[rgb]{0.00,0.00,0.00}{#1}}
\newcommand{\ControlFlowTok}[1]{\textcolor[rgb]{0.13,0.29,0.53}{\textbf{#1}}}
\newcommand{\OperatorTok}[1]{\textcolor[rgb]{0.81,0.36,0.00}{\textbf{#1}}}
\newcommand{\BuiltInTok}[1]{#1}
\newcommand{\ExtensionTok}[1]{#1}
\newcommand{\PreprocessorTok}[1]{\textcolor[rgb]{0.56,0.35,0.01}{\textit{#1}}}
\newcommand{\AttributeTok}[1]{\textcolor[rgb]{0.77,0.63,0.00}{#1}}
\newcommand{\RegionMarkerTok}[1]{#1}
\newcommand{\InformationTok}[1]{\textcolor[rgb]{0.56,0.35,0.01}{\textbf{\textit{#1}}}}
\newcommand{\WarningTok}[1]{\textcolor[rgb]{0.56,0.35,0.01}{\textbf{\textit{#1}}}}
\newcommand{\AlertTok}[1]{\textcolor[rgb]{0.94,0.16,0.16}{#1}}
\newcommand{\ErrorTok}[1]{\textcolor[rgb]{0.64,0.00,0.00}{\textbf{#1}}}
\newcommand{\NormalTok}[1]{#1}
\usepackage{graphicx,grffile}
\makeatletter
\def\maxwidth{\ifdim\Gin@nat@width>\linewidth\linewidth\else\Gin@nat@width\fi}
\def\maxheight{\ifdim\Gin@nat@height>\textheight\textheight\else\Gin@nat@height\fi}
\makeatother
% Scale images if necessary, so that they will not overflow the page
% margins by default, and it is still possible to overwrite the defaults
% using explicit options in \includegraphics[width, height, ...]{}
\setkeys{Gin}{width=\maxwidth,height=\maxheight,keepaspectratio}
\IfFileExists{parskip.sty}{%
\usepackage{parskip}
}{% else
\setlength{\parindent}{0pt}
\setlength{\parskip}{6pt plus 2pt minus 1pt}
}
\setlength{\emergencystretch}{3em}  % prevent overfull lines
\providecommand{\tightlist}{%
  \setlength{\itemsep}{0pt}\setlength{\parskip}{0pt}}
\setcounter{secnumdepth}{0}
% Redefines (sub)paragraphs to behave more like sections
\ifx\paragraph\undefined\else
\let\oldparagraph\paragraph
\renewcommand{\paragraph}[1]{\oldparagraph{#1}\mbox{}}
\fi
\ifx\subparagraph\undefined\else
\let\oldsubparagraph\subparagraph
\renewcommand{\subparagraph}[1]{\oldsubparagraph{#1}\mbox{}}
\fi

%%% Use protect on footnotes to avoid problems with footnotes in titles
\let\rmarkdownfootnote\footnote%
\def\footnote{\protect\rmarkdownfootnote}

%%% Change title format to be more compact
\usepackage{titling}

% Create subtitle command for use in maketitle
\newcommand{\subtitle}[1]{
  \posttitle{
    \begin{center}\large#1\end{center}
    }
}

\setlength{\droptitle}{-2em}

  \title{spadesCBM}
    \pretitle{\vspace{\droptitle}\centering\huge}
  \posttitle{\par}
    \author{}
    \preauthor{}\postauthor{}
      \predate{\centering\large\emph}
  \postdate{\par}
    \date{September 2018}


\begin{document}
\maketitle

\section{Overview}\label{overview}

The theme here is ``Transparency, flexibility and science improvement in
CFS carbon modelling''.

\subsection{Background}\label{background}

This is a family of SpaDES modules that emulates the science in
CBM-CFS3. It was developed on the SpaDES platform (a package in R -
\url{https://cran.r-project.org/web/packages/SpaDES/index.html}) to make
it transparent, spatial explicit and flexible. It is meant to be an
environment in which science improvements can be explored and tested.
These include links to other models for multi-use decision making and
carbon science improvements. The computing difference with CBM-CFS3 is
that the operations on the carbon pools are done by matrix
multiplications instead of simple multiplication. Being in the SpaDES
environment, it is meant to be run spatially explicitly which assumes
that the required inputs are spatially explicit.

\section{Three-module family}\label{three-module-family}

The family of modules is called from a parent module named spadesCBM.
The spadesCBM parent module calls three child modules:
spadesCBMdefaults, spadesCBMinputs, and spadesCBMcore. The code
environment is on a private repository here:
\url{https://github.com/cboisvenue/spadesCBM.git}.

The two first modules have a parsing file (the R file that has all the
functions - spadesCBMdefaultFunctions.r, spadesCBMinputsFunctions.r).
spadesCBMdefaultFunctions.r has r-language functions to build and query
the S4 object cbmData. spadesCBMinputsFunctions.r, has r-language
hashing functions and calls on the library ``CBMVolumeToBiomass''. This
library was build by Scott Morken to apply the Boudewyn et al
stand-level parameters to growth curve information for a translation
into biomass pools. This library needs to be already built before
running this module. spadesCBMcoreFunctions.r compiles the Rcpp code in
.InputObjects of spadesCBMcore.R (it has no parsing file). The compiled
code is cached in the cache directory
\texttt{cacheDir\ =\ cachePath(sim),\ env\ =\ envir(sim)}

Many more details of this three-modules family are in
G:\RES\_Work\Work\SpaDES\spadesCBM\Prezi WIN spadesCBM modules ov.exe,
which is also a working document.

\subsubsection{spadesCBMdefaults}\label{spadescbmdefaults}

This module has one event (init) and does not schedule anything else. It
requires the ``dbPath'' and ``sqlDir'' to run. This present .rmd file
creates ``dbPath'' and ``sqlDir'' (see below) and so does the
.InputObjects section of spadesCBMdefault.R so that the spadesCBMdefault
can run independently from the two other modules or from this parent
module. This module loads all the CBM-CFS3 default parameters (Canadian
defaults that is akin to the ArchiveIndex access database in CBM-CFS3).
It is read as an S4 object called cbmData. This object has the following
slot names ``turnoverRates'' (15byb13 full), ``rootParameters'' (48by7
full), ``decayParameters'' (11X6 full), ``spinupParameters''(48by4
full), ``classifierValues''(0X0), ``climate'' (48by2 full - mean annual
temp), ``spatialUnitIds'' (48by3 full), ``slowAGtoBGTransferRate''(1by1
0.006), ``biomassToCarbonRate''(1by1 0.5),``ecoIndices'' (0by0),
``spuIndices'' (0by0), ``stumpParameters'' (48by5 full),
``overmatureDeclineParameters'' (48by4 full), ``disturbanceMatrix''
(426X3 - character matrix with word descriptions of disturbances
{[}``id'' ``name'' ``description''{]}). The whole sqlite db that
contains the defaults is read/accessible via
C:\Celine\GitHub\spadesCBM\exploringCode\readInSQLiteData.r.

\subsubsection{spadesCBMinputs}\label{spadescbminputs}

This module has one event (init) and does not schedule anything else.
This module reads in information that is expected to be provided by the
user: the growth curves, the ages of the stands/pixels, links between
each stand and the growth curves, and where these stands are in Canada
(which provides a link to the default parameters read-in by the previous
module). It translates the m3/ha into the biomass pools and carbon curve
using the CBMVolumeToBiomass.dll (by Scott Morken) which is done using
the Boudewyn et al. stand-level parameters (this will eventually be a
separate module for transparency). The CBMVolumeToBiomass library needs
to already by built to run this module. This module will read-in
spatially explicit data (eventually). It reads-in min and max rotation
lengths, mean fire return interval, provided a place for regeneration
delays, and this is the place where you will tell the model where you
are (which ecozone and spatial unit). The disturbance events also go in
here (what and what date).

\subsection{spadesCBMcore}\label{spadescbmcore}

This module completes the simulations of the spadesCBM. It has five
events: spinup, postSpinup, saveSpinup, annual, and savePools, with the
saveSpinup being optional. The spinup event is in the init, and it runs
the traditional spinup similarly to CBM-CFS3 where each stand is
disturbed using the ``historicDMIDs'' and re-grown using the provided
growth curves, until the DOM pools ``fill-up'' and stabilize. A user can
set a minimun and a maximum number of rotations, and the disturbance
return interval (``minRotations'', ``maxRotations'', ``returnIntervals''
set in the spadesCBMinput module) for the spinup. Once the DOM pools
have stabilized, the spinup event growths the stand (still using the
same growth curve) to the user-provided age of that stand/pixel
(``ages'' created in spadesCBMinputs module). If spinupDebug is set to
FALSE, the spinup event provides a line for each stand with the intial
pool values to initialize the stands/pixels for the annual simulations.
These are assigned in postSpinup event. In the postSpinup event,
matrices are set up for the processes that will happen in the annual
event. The annual event is where all the processes are applied and the
savePools event is scheduled last to save a .csv of the final pool
values.

\subsubsection{DANGER with the spinupDebug
parameter}\label{danger-with-the-spinupdebug-parameter}

The spinupDebug parameter is a logical parameter defined in the metadata
of the spadesCBMcore.R module. It determines if the results from the
spinup will be saved as an external file. The default is FALSE. If this
is set to TRUE, the postSpinup event re-runs the cpp Sinup function
because the cpp Spinup function actually uses the sinpupDebug and
spits-out all the runs for all the stand/pixelss until the stabilization
of the DOM pools instead of the DOM for each stand needed to start the
annual simulation runs. This could take a long time if you have a lot of
stands/pixels. This can be changed later, it was a work around to get
these modules running.

\subsection{Known Errors carried over from
carb1}\label{known-errors-carried-over-from-carb1}

Error in Spinup(pools =
sim\(pools, opMatrix = opMatrix, constantProcesses = sim\)processes, :
Expecting a single value: {[}extent=0{]}. This is because the pooldef
values are expected to be in the .GlobalEnv due to a function inside
RCMBGrowthIncrements.cpp . That file should be changed so it is not
looking in .GlobalEnv. Current work around is to place all the pooldef
values in .GlobalEnv. This happens inside the ``.inputObjects''
function.

\section{Usage}\label{usage}

\begin{Shaded}
\begin{Highlighting}[]
\KeywordTok{library}\NormalTok{(SpaDES)}

\NormalTok{moduleDir <-}\StringTok{ }\KeywordTok{file.path}\NormalTok{(}\KeywordTok{getwd}\NormalTok{())}\CommentTok{#"C:/Celine/GitHub/spadesCBM")}
\NormalTok{inputDir <-}\StringTok{ }\KeywordTok{file.path}\NormalTok{(moduleDir,}\StringTok{"data"}\NormalTok{) }\OperatorTok\StringTok{ }\NormalTok{reproducible}\OperatorTok{::}\KeywordTok{checkPath}\NormalTok{(}\DataTypeTok{create =} \OtherTok{TRUE}\NormalTok{) }\CommentTok{#"C:/Celine/GitHub/spadesCBM/data")}
\NormalTok{outputDir <-}\StringTok{ }\KeywordTok{file.path}\NormalTok{(moduleDir,}\StringTok{"outputs"}\NormalTok{) }\CommentTok{#"C:/Celine/SpaDEScacheOutputs/outputs")}
\NormalTok{cacheDir <-}\StringTok{ }\KeywordTok{file.path}\NormalTok{(outputDir,}\StringTok{"cache"}\NormalTok{)}\CommentTok{#C:/Celine/SpaDEScacheOutputs/cache")}
\NormalTok{times <-}\StringTok{ }\KeywordTok{list}\NormalTok{(}\DataTypeTok{start =} \FloatTok{1990.00}\NormalTok{, }\DataTypeTok{end =} \FloatTok{2001.00}\NormalTok{)}
\NormalTok{parameters <-}\StringTok{ }\KeywordTok{list}\NormalTok{(}
  \DataTypeTok{spadesCBMcore =} \KeywordTok{list}\NormalTok{(}\DataTypeTok{.useCache =} \OtherTok{FALSE}\NormalTok{),}
  \DataTypeTok{spadesCBMinputs =} \KeywordTok{list}\NormalTok{(}\DataTypeTok{.useCache =} \OtherTok{FALSE}\NormalTok{),}
  \DataTypeTok{spadesCBMdefaults =} \KeywordTok{list}\NormalTok{(}\DataTypeTok{.useCache =} \OtherTok{FALSE}\NormalTok{)}
\NormalTok{)}

\NormalTok{modules <-}\StringTok{ }\KeywordTok{list}\NormalTok{(}\StringTok{"spadesCBM"}\NormalTok{)}
\NormalTok{objects <-}\StringTok{ }\KeywordTok{list}\NormalTok{(}
  \DataTypeTok{dbPath =} \KeywordTok{file.path}\NormalTok{(inputDir,}\StringTok{"cbm_defaults"}\NormalTok{,}\StringTok{"cbm_defaults.db"}\NormalTok{),}
  \DataTypeTok{sqlDir =} \KeywordTok{file.path}\NormalTok{(inputDir,}\StringTok{"cbm_defaults"}\NormalTok{)}
\NormalTok{  )}
\NormalTok{paths <-}\StringTok{ }\KeywordTok{list}\NormalTok{(}
  \DataTypeTok{cachePath =}\NormalTok{ cacheDir,}
  \DataTypeTok{modulePath =}\NormalTok{ moduleDir,}
  \DataTypeTok{inputPath =}\NormalTok{ inputDir,}
  \DataTypeTok{outputPath =}\NormalTok{ outputDir}
\NormalTok{  )}

\NormalTok{spadesCBMSim <-}\StringTok{ }\KeywordTok{simInit}\NormalTok{(}\DataTypeTok{times =}\NormalTok{ times, }\DataTypeTok{params =}\NormalTok{ parameters, }
                    \DataTypeTok{modules =}\NormalTok{ modules,}
                 \DataTypeTok{objects =}\NormalTok{ objects, }\DataTypeTok{paths =}\NormalTok{ paths)}

\CommentTok{#spadesCBMout <- spades(spadesCBMSim,debug=TRUE)}
\NormalTok{spadesCBMddist <-}\StringTok{ }\KeywordTok{spades}\NormalTok{(spadesCBMSim,}\DataTypeTok{debug=}\OtherTok{TRUE}\NormalTok{)}
\end{Highlighting}
\end{Shaded}

\section{Noteworthy objects}\label{noteworthy-objects}

\begin{itemize}
\tightlist
\item
  pooldef: names of all the pools that also have a memory spot saved in
  teh .Globals right now
\item
  allMatrices: return the matrix ids of the loaded matrices
\item
  decayRates: matrix 48X12 - holds the decay rates for all the dom pools
  (11 of them) per spatial unit. Where is this created? How does it link
  to the ``decay\_parameter'' table in cbmData?
\item
  carbonCurve: this is the result from using all the growth curves and
  ``translating'' them into biomass compartments using SMorken's
  CBMVolumeToBiomass library and its function VolumeToBiomassConvert().
  This is one of the components that will need to be made into an
  independent module and more transparent
\item
  growth\_increments: not convinced we need this to be in the sim
  environment\ldots{}but, it would be good to have this easily visible.
\item
  processes: this is a list of ``matrixHash''-ed (R function in
  spadesCBMinputsFunctions.r) of domDecay, slowDecay, slowMixing,
  domTurnover, bioTurnover, disturbanceMatrices.
\end{itemize}

\subsection{Plotting}\label{plotting}

Nothing for now.

\subsection{Saving}\label{saving}

Pools at the end of all runs via the savePools event of the
spadesCBMcore module.

\section{Data dependencies}\label{data-dependencies}

\subsection{Input data}\label{input-data}

Working on using the spatial data from SK. These are stored here
C:\Celine\Syndocs\RES\_Work\Work\SpaDES\SK\_data. Eventually external
sources of growth and yield and forest inventory can be used.

\subsection{Output data}\label{output-data}

See G:\RES\_Work\Work\SpaDES\spadesCBM\Prezi WIN spadesCBM modules
ov.exe for now

\section{Links to other modules}\label{links-to-other-modules}

This will be linked to the SpaDES caribou models, to the SpaDES version
of LANDIS-II, and to any other available SpaDES modules available
(growth and yield?).


\end{document}
