\documentclass[]{article}
\usepackage{lmodern}
\usepackage{amssymb,amsmath}
\usepackage{ifxetex,ifluatex}
\usepackage{fixltx2e} % provides \textsubscript
\ifnum 0\ifxetex 1\fi\ifluatex 1\fi=0 % if pdftex
  \usepackage[T1]{fontenc}
  \usepackage[utf8]{inputenc}
\else % if luatex or xelatex
  \ifxetex
    \usepackage{mathspec}
  \else
    \usepackage{fontspec}
  \fi
  \defaultfontfeatures{Ligatures=TeX,Scale=MatchLowercase}
\fi
% use upquote if available, for straight quotes in verbatim environments
\IfFileExists{upquote.sty}{\usepackage{upquote}}{}
% use microtype if available
\IfFileExists{microtype.sty}{%
\usepackage{microtype}
\UseMicrotypeSet[protrusion]{basicmath} % disable protrusion for tt fonts
}{}
\usepackage[margin=1in]{geometry}
\usepackage{hyperref}
\hypersetup{unicode=true,
            pdftitle={spadesCBM},
            pdfborder={0 0 0},
            breaklinks=true}
\urlstyle{same}  % don't use monospace font for urls
\usepackage{color}
\usepackage{fancyvrb}
\newcommand{\VerbBar}{|}
\newcommand{\VERB}{\Verb[commandchars=\\\{\}]}
\DefineVerbatimEnvironment{Highlighting}{Verbatim}{commandchars=\\\{\}}
% Add ',fontsize=\small' for more characters per line
\usepackage{framed}
\definecolor{shadecolor}{RGB}{248,248,248}
\newenvironment{Shaded}{\begin{snugshade}}{\end{snugshade}}
\newcommand{\KeywordTok}[1]{\textcolor[rgb]{0.13,0.29,0.53}{\textbf{#1}}}
\newcommand{\DataTypeTok}[1]{\textcolor[rgb]{0.13,0.29,0.53}{#1}}
\newcommand{\DecValTok}[1]{\textcolor[rgb]{0.00,0.00,0.81}{#1}}
\newcommand{\BaseNTok}[1]{\textcolor[rgb]{0.00,0.00,0.81}{#1}}
\newcommand{\FloatTok}[1]{\textcolor[rgb]{0.00,0.00,0.81}{#1}}
\newcommand{\ConstantTok}[1]{\textcolor[rgb]{0.00,0.00,0.00}{#1}}
\newcommand{\CharTok}[1]{\textcolor[rgb]{0.31,0.60,0.02}{#1}}
\newcommand{\SpecialCharTok}[1]{\textcolor[rgb]{0.00,0.00,0.00}{#1}}
\newcommand{\StringTok}[1]{\textcolor[rgb]{0.31,0.60,0.02}{#1}}
\newcommand{\VerbatimStringTok}[1]{\textcolor[rgb]{0.31,0.60,0.02}{#1}}
\newcommand{\SpecialStringTok}[1]{\textcolor[rgb]{0.31,0.60,0.02}{#1}}
\newcommand{\ImportTok}[1]{#1}
\newcommand{\CommentTok}[1]{\textcolor[rgb]{0.56,0.35,0.01}{\textit{#1}}}
\newcommand{\DocumentationTok}[1]{\textcolor[rgb]{0.56,0.35,0.01}{\textbf{\textit{#1}}}}
\newcommand{\AnnotationTok}[1]{\textcolor[rgb]{0.56,0.35,0.01}{\textbf{\textit{#1}}}}
\newcommand{\CommentVarTok}[1]{\textcolor[rgb]{0.56,0.35,0.01}{\textbf{\textit{#1}}}}
\newcommand{\OtherTok}[1]{\textcolor[rgb]{0.56,0.35,0.01}{#1}}
\newcommand{\FunctionTok}[1]{\textcolor[rgb]{0.00,0.00,0.00}{#1}}
\newcommand{\VariableTok}[1]{\textcolor[rgb]{0.00,0.00,0.00}{#1}}
\newcommand{\ControlFlowTok}[1]{\textcolor[rgb]{0.13,0.29,0.53}{\textbf{#1}}}
\newcommand{\OperatorTok}[1]{\textcolor[rgb]{0.81,0.36,0.00}{\textbf{#1}}}
\newcommand{\BuiltInTok}[1]{#1}
\newcommand{\ExtensionTok}[1]{#1}
\newcommand{\PreprocessorTok}[1]{\textcolor[rgb]{0.56,0.35,0.01}{\textit{#1}}}
\newcommand{\AttributeTok}[1]{\textcolor[rgb]{0.77,0.63,0.00}{#1}}
\newcommand{\RegionMarkerTok}[1]{#1}
\newcommand{\InformationTok}[1]{\textcolor[rgb]{0.56,0.35,0.01}{\textbf{\textit{#1}}}}
\newcommand{\WarningTok}[1]{\textcolor[rgb]{0.56,0.35,0.01}{\textbf{\textit{#1}}}}
\newcommand{\AlertTok}[1]{\textcolor[rgb]{0.94,0.16,0.16}{#1}}
\newcommand{\ErrorTok}[1]{\textcolor[rgb]{0.64,0.00,0.00}{\textbf{#1}}}
\newcommand{\NormalTok}[1]{#1}
\usepackage{graphicx,grffile}
\makeatletter
\def\maxwidth{\ifdim\Gin@nat@width>\linewidth\linewidth\else\Gin@nat@width\fi}
\def\maxheight{\ifdim\Gin@nat@height>\textheight\textheight\else\Gin@nat@height\fi}
\makeatother
% Scale images if necessary, so that they will not overflow the page
% margins by default, and it is still possible to overwrite the defaults
% using explicit options in \includegraphics[width, height, ...]{}
\setkeys{Gin}{width=\maxwidth,height=\maxheight,keepaspectratio}
\IfFileExists{parskip.sty}{%
\usepackage{parskip}
}{% else
\setlength{\parindent}{0pt}
\setlength{\parskip}{6pt plus 2pt minus 1pt}
}
\setlength{\emergencystretch}{3em}  % prevent overfull lines
\providecommand{\tightlist}{%
  \setlength{\itemsep}{0pt}\setlength{\parskip}{0pt}}
\setcounter{secnumdepth}{0}
% Redefines (sub)paragraphs to behave more like sections
\ifx\paragraph\undefined\else
\let\oldparagraph\paragraph
\renewcommand{\paragraph}[1]{\oldparagraph{#1}\mbox{}}
\fi
\ifx\subparagraph\undefined\else
\let\oldsubparagraph\subparagraph
\renewcommand{\subparagraph}[1]{\oldsubparagraph{#1}\mbox{}}
\fi

%%% Use protect on footnotes to avoid problems with footnotes in titles
\let\rmarkdownfootnote\footnote%
\def\footnote{\protect\rmarkdownfootnote}

%%% Change title format to be more compact
\usepackage{titling}

% Create subtitle command for use in maketitle
\newcommand{\subtitle}[1]{
  \posttitle{
    \begin{center}\large#1\end{center}
    }
}

\setlength{\droptitle}{-2em}

  \title{spadesCBM}
    \pretitle{\vspace{\droptitle}\centering\huge}
  \posttitle{\par}
    \author{}
    \preauthor{}\postauthor{}
      \predate{\centering\large\emph}
  \postdate{\par}
    \date{September 2018}


\begin{document}
\maketitle

\section{Overview}\label{overview}

The theme here is ``Transparency, flexibility and science improvement in
CFS carbon modelling''.

\subsection{Background}\label{background}

This is a family of SpaDES modules that emulates the science in CBM-CFS3
(Kurz et al.2009). It was developed on the SpaDES platform (a package in
R - \url{https://cran.r-project.org/web/packages/SpaDES/index.html}) to
make it transparent, spatial explicit and flexible. Spades is a Spatial
Discrete Event Simulator. It is an R-package that functions as a
scheduler through space and time. Being an R-based platform, is makes
modelling transparent and accessible to a large community of researchers
across disciplines. The family of modules, spadesCBM, is meant to be an
environment in which science improvements can be explored and tested.
These include links to other models for multi-use decision making and
carbon science improvements. More information on SpaDES and other openly
available SpaDES modules can be found here
\url{http://spades.predictiveecology.org/}.

The only difference between spadesCBM and CBM-CFS3 is that spadesCBM
modifies the carbon pools via matrix multiplications instead of simple
multiplication which CBM-CFS3 does. Being in the SpaDES environment, it
is meant to be run spatially explicitly (CBM-CFS3 is not) which assumes
that the required inputs are spatially explicit. Knowledge of the SpaDES
structure would help an R-knowledgable user to manipulate simulations
but is not necessary to run the current simulations. The code-chunk in
this document will run simulations for managed forests in Saskatchewan,
Canada. Prior knowledge of CBM-CFS3 would also help users understand the
structure of these modules, the default parameters used, but is not
necessary to run simulations. All modules being written in R and the
publically available description of the SpaDES R-package imply that any
R-user can learn how to run these modules and simulate carbon on a
landscape. In this document, I describe all three modules necessary for
simulations using spadesCBM. The history of how these modules were
developed can be find in spadesCBMhistory.Rmd.

\section{Three-module family}\label{three-module-family}

The family of modules is called from a parent module named spadesCBM.
The spadesCBM parent module calls three child modules:
spadesCBMdefaults, spadesCBMinputs, and spadesCBMcore. The code
environment is on a private repository here:
\url{https://github.com/cboisvenue/spadesCBM.git}. You need permission
and a github account to acess this and for now, please don't distribute
this code.

The two first modules each have a parsing file (an R file that has all
the functions - spadesCBMdefaultFunctions.r,
spadesCBMinputsFunctions.r). The spadesCBMdefaultFunctions.r has
r-language functions to build and query the S4 object cbmData, while
spadesCBMinputsFunctions.r, has r-language hashing functions and calls
on the library ``CBMVolumeToBiomass''. This library was build by Scott
Morken to apply the Boudewyn \emph{et al.}(2007) stand-level parameters,
to growth curve information (user provided) for a translation into
biomass pools. This library needs to be already built before running
this module. The code for this library is on this private github
repository (\url{https://github.com/smorken/CBMVolumeToBiomass}). In
SpaDES, parsing files get compiled when a simlist gets created. The
spadesCBMcore module (spadesCBMcore.r) compiles the Rcpp code in
.InputObjects and has no parsing file.

Many more details of this three-modules family are in this prezi
G:\RES\_Work\Work\SpaDES\spadesCBM\Prezi WIN spadesCBM modules ov.exe,
which is also a working document.

\subsubsection{spadesCBMdefaults}\label{spadescbmdefaults}

This module loads all the CBM-CFS3 default parameters (Canadian defaults
that is akin to the ArchiveIndex access database in CBM-CFS3). These
parameters are then stored in an S4 object called cbmData and accessed
throughout the simulations. This object has the following slot names
``turnoverRates'' (15byb13 full), ``rootParameters'' (48by7 full),
``decayParameters'' (11X6 full), ``spinupParameters''(48by4 full),
``classifierValues''(0X0), ``climate'' (48by2 full - mean annual temp),
``spatialUnitIds'' (48by3 full), ``slowAGtoBGTransferRate''(1by1 0.006),
``biomassToCarbonRate''(1by1 0.5),``ecoIndices'' (0by0), ``spuIndices''
(0by0), ``stumpParameters'' (48by5 full),
``overmatureDeclineParameters'' (48by4 full), ``disturbanceMatrix''
(426X3 - character matrix with word descriptions of disturbances
{[}``id'' ``name'' ``description''{]}). The whole sqlite db that
contains the defaults is stored in this RStudio project
(spadesCBM.Rproj) in the data folder \spadesCBM\data\cbm\_defaults. The
script spadesCBM\exploringCode\readInSQLiteData.r further explores the
data in the sql database. \emph{All parameters used in these simulations
and the general canadian defaults, are searcheable with common R
functionality}. In the SpaDES environment, this module has one event
(init) and does not schedule anything else. It requires the ``dbPath''
and ``sqlDir'' to run. This present .rmd file creates ``dbPath'' and
``sqlDir'' (see below) and so does the .InputObjects section of
spadesCBMdefault.R so that the spadesCBMdefault can run independently
from the two other modules or from this parent module.

\subsubsection{spadesCBMinputs}\label{spadescbminputs}

This module reads in information that is expected to be provided by the
user similarly to CBM-CFS3: the growth curves, the ages of the
stands/pixels, links between each stand and the growth curves, and where
these stands are in Canada (which provides a link to the default
parameters read-in by the previous module). It translates the m3/ha of
the growth curves into the biomass pools for stem wood, bark, branches
and foliage using the CBMVolumeToBiomass library (by Scott Morken). This
library is based on the stand-level parameters of Boudewyn \emph{et
al.}(2007). The volume to biomass translations will eventually be a
separate module to increase transparency of this process and permit the
use of other biomass curves (example from LandR-biomass, a SpaDES module
of the vegetation dynamics model LANDIS-II). This module reads-in
spatially explicit data (in this case: age, leading species,
productivity level, cbm\_default spatial unit, growth curve
identification for each pixel) which define the unique pixel groups for
modelling. It also reads-in min and max rotation lengths, mean fire
return interval, and provides a place for regeneration delays.
Disturbance events must also be spatially explicit and will redefine the
pixel groups post-disturbance. Disturbances information needs to specify
what type of disturbance using the cbm\_default disturbance matrix
numbers (see cbmData), the year of disturbance and the pixel or stands
disturbed. This module has one event (init) and does not schedule
anything else.

\subsection{spadesCBMcore}\label{spadescbmcore}

This module completes the simulations of the spadesCBM. It has five
SpaDES-events: spinup, postSpinup, saveSpinup, annual, and savePools,
with the saveSpinup being optional. The spinup event is the ``init''
event run by default in SpaDES modules. The event ``spinup'' runs the
traditional spinup of CBM-CFS3: where each stand/pixel is disturbed
using the disturbance specified in ``historicDMIDs'' and re-grown using
the provided five aboveground biomass pools, until the dead organic
matter (DOM) pools values stabilize. A user can set a minimun and a
maximum number of rotations, and the disturbance return interval
(``minRotations'', ``maxRotations'', ``returnIntervals'') for the
spinup. Once the DOM pools have stabilized, the spinup event grows the
stand (still using the same growth curve) to the user-provided age of
that stand/pixel (``ages'' created in spadesCBMinputs module). If
spinupDebug is set to FALSE, the spinup event provides a line for each
stand with the intial pool values to initialize the stands/pixels for
the annual simulations. These are assigned in the postSpinup event. In
the postSpinup event, matrices are set up for the processes that will
happen in the annual event. The event spinupDebug was put in place to
explore the results of the spinup and if TRUE, it saves \textbf{ALL} the
disturbed-grow cycles. Please refer to the next paragraph for warnings
about this event. The annual event is where all the processes are
applied. These include annual growth, turnover, overmature decline,
decay, and disturbances. The event ``savePools'' is scheduled last. It
currently creates a ``.csv'' file (output1stand.csv) that contains the
carbon pool values for each unique stand/pixel type at the end of each
simulation year. Outputs can modified as needed.

\paragraph{DANGER with the spinupDebug
parameter}\label{danger-with-the-spinupdebug-parameter}

The spinupDebug parameter is a logical parameter defined in the metadata
of the spadesCBMcore.R module. It determines if the results from the
spinup will be saved as an external file. The default is FALSE. If this
is set to TRUE, the postSpinup event re-runs the cpp Sinup function
because the cpp Spinup function actually uses the sinpupDebug and
spits-out all the runs for all the stand/pixels until the stabilization
of the DOM pools instead of the DOM for each stand needed to start the
annual simulation runs. This could take a long time if you have a lot of
stands/pixels. This can be changed later, it was a work around to get
these modules running.

\paragraph{Known Errors to eventually fix in the Rcpp
scripts}\label{known-errors-to-eventually-fix-in-the-rcpp-scripts}

Error in Spinup(pools =
sim\(pools, opMatrix = opMatrix, constantProcesses = sim\)processes, :
Expecting a single value: {[}extent=0{]}. This is because the pooldef
values are expected to be in the .GlobalEnv due to a function inside
RCMBGrowthIncrements.cpp . That file should be changed so it is not
looking in .GlobalEnv. Current work around is to place all the pooldef
values in .GlobalEnv. This happens inside the ``.inputObjects''
function.

\section{Simulations}\label{simulations}

\begin{Shaded}
\begin{Highlighting}[]
\KeywordTok{library}\NormalTok{(SpaDES)}

\NormalTok{moduleDir <-}\StringTok{ }\KeywordTok{file.path}\NormalTok{(}\KeywordTok{getwd}\NormalTok{())}\CommentTok{#"C:/Celine/GitHub/spadesCBM")}
\NormalTok{inputDir <-}\StringTok{ }\KeywordTok{file.path}\NormalTok{(moduleDir,}\StringTok{"data"}\NormalTok{) }\OperatorTok\StringTok{ }\NormalTok{reproducible}\OperatorTok{::}\KeywordTok{checkPath}\NormalTok{(}\DataTypeTok{create =} \OtherTok{TRUE}\NormalTok{) }\CommentTok{#"C:/Celine/GitHub/spadesCBM/data")}
\NormalTok{outputDir <-}\StringTok{ }\KeywordTok{file.path}\NormalTok{(moduleDir,}\StringTok{"outputs"}\NormalTok{) }\CommentTok{#"C:/Celine/SpaDEScacheOutputs/outputs")}
\NormalTok{cacheDir <-}\StringTok{ }\KeywordTok{file.path}\NormalTok{(outputDir,}\StringTok{"cache"}\NormalTok{)}\CommentTok{#C:/Celine/SpaDEScacheOutputs/cache")}
\NormalTok{times <-}\StringTok{ }\KeywordTok{list}\NormalTok{(}\DataTypeTok{start =} \FloatTok{1990.00}\NormalTok{, }\DataTypeTok{end =} \FloatTok{2005.00}\NormalTok{)}
\NormalTok{parameters <-}\StringTok{ }\KeywordTok{list}\NormalTok{(}
  \DataTypeTok{spadesCBMcore =} \KeywordTok{list}\NormalTok{(}\DataTypeTok{.useCache =} \OtherTok{FALSE}\NormalTok{),}
  \DataTypeTok{spadesCBMinputs =} \KeywordTok{list}\NormalTok{(}\DataTypeTok{.useCache =} \OtherTok{FALSE}\NormalTok{),}
  \DataTypeTok{spadesCBMdefaults =} \KeywordTok{list}\NormalTok{(}\DataTypeTok{.useCache =} \OtherTok{FALSE}\NormalTok{)}
\NormalTok{)}

\NormalTok{modules <-}\StringTok{ }\KeywordTok{list}\NormalTok{(}\StringTok{"spadesCBM"}\NormalTok{)}
\NormalTok{objects <-}\StringTok{ }\KeywordTok{list}\NormalTok{(}
  \DataTypeTok{dbPath =} \KeywordTok{file.path}\NormalTok{(inputDir,}\StringTok{"cbm_defaults"}\NormalTok{,}\StringTok{"cbm_defaults.db"}\NormalTok{),}
  \DataTypeTok{sqlDir =} \KeywordTok{file.path}\NormalTok{(inputDir,}\StringTok{"cbm_defaults"}\NormalTok{)}
\NormalTok{  )}
\NormalTok{paths <-}\StringTok{ }\KeywordTok{list}\NormalTok{(}
  \DataTypeTok{cachePath =}\NormalTok{ cacheDir,}
  \DataTypeTok{modulePath =}\NormalTok{ moduleDir,}
  \DataTypeTok{inputPath =}\NormalTok{ inputDir,}
  \DataTypeTok{outputPath =}\NormalTok{ outputDir}
\NormalTok{  )}

\NormalTok{spadesCBMSim <-}\StringTok{ }\KeywordTok{simInit}\NormalTok{(}\DataTypeTok{times =}\NormalTok{ times, }\DataTypeTok{params =}\NormalTok{ parameters, }
                    \DataTypeTok{modules =}\NormalTok{ modules,}
                 \DataTypeTok{objects =}\NormalTok{ objects, }\DataTypeTok{paths =}\NormalTok{ paths)}

\NormalTok{spadesCBMout <-}\StringTok{ }\KeywordTok{spades}\NormalTok{(spadesCBMSim,}\DataTypeTok{debug=}\OtherTok{TRUE}\NormalTok{)}
\end{Highlighting}
\end{Shaded}

\section{Noteworthy objects and periferal
functions}\label{noteworthy-objects-and-periferal-functions}

\paragraph{Objects}\label{objects}

\begin{itemize}
\tightlist
\item
  pooldef: names of all the pools that also have a memory spot saved in
  the .Globals right now
\item
  allMatrices: return the matrix ids of the loaded matrices
\item
  decayRates: matrix 48X12 - holds the decay rates for all the dom pools
  (11 of them) per spatial unit. Where is this created? How does it link
  to the ``decay\_parameter'' table in cbmData?
\item
  carbonCurve: this is the result from using all the growth curves and
  ``translating'' them into biomass compartments using SMorken's
  CBMVolumeToBiomass library and its function VolumeToBiomassConvert().
  This is one of the components that will need to be made into an
  independent module and more transparency
\item
  growth\_increments: not convinced we need this to be in the sim
  environment\ldots{}but, it would be good to have this easily visible.
\item
  processes: this is a list of ``matrixHash''-ed (R function in
  spadesCBMinputsFunctions.r) of domDecay, slowDecay, slowMixing,
  domTurnover, bioTurnover, disturbanceMatrices.
\end{itemize}

\paragraph{Functions}\label{functions}

A series of functions were built to help get details on CBM-CFS3 default
disturbances. There are spuDist(), histDist(), seeDist() and simDist().
They are stored here \spadesCBM\exploringCode\spadesCBMextraFunctions.r.

The spuDist() function identifies the ID number (CBM-CFS3 legacy) that
are possible in the specific spatial unit you are in. You give the
spatial units id(s) you are targetting (mySpu) and it give you the
disturbance matrix id(s) that are possible/default in that specific spu
and a descriptive name of that disturbance matrix. It returns a
data.frame. This is useful for identifying the lastPass disturbance
possibilities in a specific spu, or the right clearcut or fire for that
specific spu.

Historical disturbances in CBM-CFS3 are used for ``filling-up'' the
soil-related carbon pools. Boudewyn et al. (2007) translate the m3/ha
curves into biomass per ha in each of four pools: total biomass for stem
wood, total biomass for bark, total biomass for branches and total
biomass for foliage. Biomass in coarse and fine roots, in aboveground-
and belowground- very-fast, -fast, -slow, in medium-soil, and in snags
still need to be estimated. In all spatial units in Canada, the
historical disturbance is set to fire. A stand-replacing fire
disturbance is used in a disturb-grow cycle, where stands are disturbed
and regrown with turnover, overmature, and decay processes, until the
dead organic matter pools biomass values stabilise (+ or - 10\% I think,
that is in the Rcpp-RCMBGrowthIncrements.cpp). This function histDist(),
identifies the stand-replacing wildfire disturbance in each spatial
unit. By default the most recent is selected, but the user can change
that. As per spuDist, you need to specify your spu.

You give the seeDist() function one or more disturbance matrix id, and
it will return the descriptive name of the disturbance, the source
pools, the sink pools, and the proportions transferred. It returns a
list of data frames, one data.frame per disturbance matrix id, similarly
to simDist().

simDist() is an R function that requires a simulation list (from the
SpaDES functions simList() or spades()) and returns a list of
data.frames. Each data.frame has the descriptive name of a disturbance
used in the simulation. Each data.frame has the disturbance matrix
identification number from cbm\_defaults, the pool from which carbon is
taken (source pools) in this specific disturbance, the pools into which
carbon goes, and the proportion in which the carbon-transfers are
completed.

\section{Data dependencies}\label{data-dependencies}

These three modules require the cbm\_defaults data and user defined
inputs. Eventually, some of the inputs may come from other modules
(growth and yield modules, biomass dynamics modules, etc.).

\subsection{Input data}\label{input-data}

The current example simulations are for managed forests in Saskatchewan.
Simulations use the same data as in Boisvenue \emph{et al.} (2016).
These data are here: \spadesCBM\data\forIan\SK\_data. Eventually, the
growth and yield curve development that was done using the PSP in
Saskatchewan will be an explicit module that could be replaced by
external sources of growth and yield. Similarly, all user-defined inputs
will eventually be able to be sourced elsewhere.

\subsection{Output data}\label{output-data}

For now, the stand carbon pools at the end of each simluation year can
be found here: \spadesCBM\outputs.

\section{Links to other modules}\label{links-to-other-modules}

This will be linked to the SpaDES caribou models, to the SpaDES version
of LANDIS-II (Land-R-biomass), and to any other available SpaDES modules
available (growth and yield?).

\section{Other important information I don't know where to put
yet}\label{other-important-information-i-dont-know-where-to-put-yet}

\textbf{MasterRater}: this is the raster that we use to show results of
simulations. In teh present SK project, It is created in
spadesCBMinputs.R (line 184) and for these SK simulations is it the
ldSpsRaster, so it has 0 for no-species pixels (we can't model these),
and 5 0 3 4 6 7. The species code table is: rasterValue species\_code 0
No\_Spp 1 Abie\_Bal 2 Popu\_Bal 3 Pice\_Mar 4 Pinu\_Ban 5 Popu\_Tre 6
Betu\_Pap 7 Pice\_Gla

\textbf{standIndex}: this comes out of the cpp functions. It is always
on the right-hand side of the of assignments in the R scripts.
Presently, to make sure that standIndex matches PixelGroupID, rows are
orderd by PixelGroupID before going into cpp functions.

\textbf{level3DT}: this is the data.table that is fed into the Rcpp
functions (matrix operations of annual processes and disturbances). Each
line represents a PixelGroupID, i.e.~a group of unique pixels. It has
unique combinations of growth\_curve\_component\_id, rasterSps,
Productivity, spatial\_unit\_id and PixelGroupID, the unique identifier
for the group. level3DT populates the vectors needed for the Rcpp
functions (sim\(ages, sim\)gcids, sim\$spatialUnits, etc.).Post-spin-up,
this table is remade annually because HERE
ages,rasterSps,Productivity,spatial\_unit\_id

\textbf{spatialDT}: this has all the modelled pixels numbered from 1 to
npixels (test area 1 347 529 modelled pixels out of 3 705 000 ncells in
the masterRaster). The
sim\(pixelKeep, a data.table that has rowOrder from sim\)spatialDT\(rowOrder as the 1st column, and a column for each simulated year starting with the level3DT PixelGroupID our of the spinUp (PixelGroupID0), followed by one per simulated year (e.g., PixelGroupID1990). sim\)pixelKeep
gets build 1 column at a time for each year of the simulation. spatialDT
is rebuilt every year because disturbances change ages, one of the
unique identifiers. spatialDT is used to add a column called events that
is extractor each year from the disturbance rasters stack
(sim\$disturbanceRasters).

\textbf{disturbance rasters}: Disturbance rasters, from White and
Wulder, are use for disturbances in these simulations. Their names are
read in teh spadesCBMinputs module as a list. In the annual event in
spadesCBMcore, the raster for that year is read, and an events column is
added to the spatialDT using just the non-zero pixels from the
masterRaster (in this case pixels != 0). The PixelGrouID is then
calculated using unique combinations of sparial\_unit\_id,
growth\_curve\_component\_id, ages, and events in the spatialDT.
\emph{NOTE} the first spatialDT, used in the spinUp event does not have
an events column.

\section{Refences}\label{refences}

\begin{enumerate}
\def\labelenumi{\arabic{enumi}.}
\tightlist
\item
  Kurz, W.A.; Dymond, C.C.; White, T.M.; Stinson, G.; Shaw, C.H.;
  Rampley, G.J.; Smyth, C.; Simpson, B.N.; Neilson, E.T.; Trofymow,
  J.A., et al. CBM-CFS3: A model of carbon-dynamics in forestry and
  land-use change implementing IPCC standards. Ecological Modelling
  2009, 220, 480-504.
\item
  Boudewyn, P.; Song, X.; Magnussen, S.; Gillis, M.D.~Model-based,
  volume-to-biomass conversion for forested and vegetated land in
  Canada; BC-X-411; Natural Resources Canada: Victoria, BC, 2007.
\item
  Boisvenue, C.; Smiley, B.P.; White, J.C.; Kurz, W.A.; Wulder,
  M.A.~Improving carbon monitoring and reporting in forests using
  spatially-explicit information. Carbon Balance and Management 2016,
  11, 23, \url{doi:10.1186/s13021-016-0065-6}.
\end{enumerate}


\end{document}
